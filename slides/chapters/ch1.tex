%  \documentclass[11pt]{beamer}
% \usepackage{../settings}
% \usepackage{lipsum} % Per generare testo di esempio
% \usepackage{tcolorbox}

%  \begin{document}

% \title{DISTRIBUTION SHIFT}
% \subtitle{A Study on Their Effects on Statistical Models and Strategies for Mitigation}
% \date{}
% \author{Andrea Spinelli, Giacomo Amerio,\\Giovanni Lucarelli, Tommaso Piscitelli}
% \institute{University of Trieste}
% \titlegraphic{\hfill\includegraphics[height=1.5cm]{assets/logo100_orizzontale.pdf}}



\section{Introduction}


\begin{frame}{Aims of project}
	\begin{columns} % Inizia l'ambiente columns
		% Colonna sinistra
		\begin{column}{0.5\textwidth}
			\textbf{Models:}
			\begin{tcolorbox}[colframe=blue!50!black, colback=blue!5, coltitle=black, sharp corners]
				\begin{itemize}
					\item Logistic Regression
					\item Random Forest
					\item Gradient Boosting
					\item XGBoost
					% Aggiungi altri modelli qui
				\end{itemize}
			\end{tcolorbox}
		\end{column}
		
		% Colonna destra
		\vspace{0.5cm}
		\begin{column}{0.5\textwidth}
			%	\vspace{0.2cm}
			\textbf{Roadmap:}
			\begin{itemize}
				\item Creation of synthetic data and data affected by shift
				\item Evaluation of model performance on the data
				%	\item Selection of the best performing model
				\item Identification of improvement strategies (R.A.W.)
			\end{itemize}
		\end{column}
	\end{columns} % Chiudi l'ambiente columns
\end{frame}


\begin{frame}{Dataset shift}
	\begin{itemize}
		\item \textbf{Dataset shift} is a common problem in machine learning.
		
		\item It occurs when the distribution of the training data differs from the distribution of the test data.
		
		\item This can lead to a decrease in the performance of the model.
	\end{itemize}
	The two most common and well-studied causes of Dataset shift are:
	\begin{itemize}
		\item \textbf{Sample selection bias} (e.g. Economic studies) %  Clinical trials
		
		\item \textbf{Non stationary environments}%(Content recommendation system of Netflix)
	\end{itemize}
\end{frame} 


\begin{frame}{Covariate shift}
	%Can be formally defined as follows.
	Consider a target variable \( X \) and a response variable \( Y \). Let \( P_{\text{tra}} \) denote the probability distribution of the training data and \( P_{\text{tst}} \) denote the probability distribution of the test data. A \textbf{covariate shift} occurs when:  
	\vspace{0.3cm}
	\[	P_{\text{tra}}(Y \mid X) = P_{\text{tst}}(Y \mid X) \quad \text{but} \quad P_{\text{tra}}(X) \neq P_{\text{tst}}(X)	\]
	
	\begin{figure}[H]
		\centering
		\includegraphics[width=9cm]{../assets/immagine.png}
		\label{fig:immagine}
	\end{figure}
\end{frame}

\begin{frame}{Example}
	Consider a model designed to distinguish between images of cats and dogs:
	
	\vspace{0.5cm}
	\begin{minipage}[t]{0.55\textwidth} % 45% della larghezza per l'immagine
		\begin{flushleft}
			\textbf{\textcolor{blue}{Training set:}}
		\end{flushleft}
		\centering
		\includegraphics[width=5cm]{../assets/cat-dog-train.png}
		\label{fig:cani-gatti}
		\vspace{-0.2cm}
		\begin{flushleft}
			\textbf{\textcolor{blue}{Test set:}}
		\end{flushleft}
		\centering
		\includegraphics[width=5cm]{../assets/cat-dog-test.png}
		\label{fig:cani-gatti}
		
	\end{minipage}
	\hfill
	\begin{minipage}[t]{0.35\textwidth} % 45% della larghezza per la didascalia
		\vspace{0.2cm}
		\begin{tcolorbox}[colframe=blue!50!black, colback=blue!5, coltitle=black, sharp corners]
			Model will not accurately distinguish between cats and dogs because the feature distribution will differ.
		\end{tcolorbox}
	\end{minipage}
\end{frame}

\begin{frame}{Inaccurate Model}
	\begin{figure}[H]
		\centering
		\includegraphics[width=6cm]{../assets/inacurate_model.png}
		\label{fig:inacurate_model}
	\end{figure}
	Changes in the features distribution can significantly impact the model's accuracy.
\end{frame}


%  \end{document}