\chapter{Data Generation}

To evaluate the performance of the proposed models under varying degrees of covariate distribution shift, we simulated a training dataset and multiple test datasets, each exhibiting a distinct degree of shift. 
This setup enables a comprehensive assessment of the models' generalization capabilities.

The training dataset consists of three features, denoted as $X_1$, $X_2$, and $X_3$, and a binary target variable $Y$. The features were generated from a multivariate normal distribution $\mathcal{N}(\boldsymbol{\mu}, \boldsymbol{\Sigma})$, where each element of the mean vector $\boldsymbol{\mu}i$ was sampled from a uniform distribution $\mathcal{U}_{[0, 1]}$, and the elements of the covariance matrix $[\boldsymbol{\Sigma}]_{i,j}$ were sampled from $\mathcal{U}_{[-1, 1]}$. To ensure the validity of the covariance matrix, after the generation, it was then transformed into a symmetric positive semi-definite matrix.
The target variable $Y$ is a binary variable with values in $\{0, 1\}$. It was generated by first constructing a second-order polynomial model with all possible interaction terms and random coefficients drawn from $\mathcal{U}_{[-1, 1]}$. The output of this polynomial was then transformed using the standard logistic function to produce probabilities, which were thresholded to determine the binary values of $Y$.

A fully shifted test dataset was generated from a new multivariate normal distribution $\mathcal{N}(\boldsymbol{\mu}_{0.05}, \boldsymbol{\Sigma}_s)$. Here, $\boldsymbol{\mu}_{0.05}$ represents a mean vector centered at the 5th percentile of the original $\boldsymbol{\mu}$, and $\boldsymbol{\Sigma}_s$ is a covariance matrix distinct from $\boldsymbol{\Sigma}$. This shift was deliberately designed to focus on a region of the sample space with limited representation in the training data, thereby introducing a significant covariate distribution shift. The target variable $Y$ for this dataset was generated in the same manner as in the training dataset.

To assess model performance across a continuum of distribution shifts, we created a series of datasets using statistical mixtures of the training dataset and the fully shifted dataset. Specifically, we defined ten datasets, $\mathcal{D}_p$, where $p \in \{0.0, 0.1, \ldots, 1.0\}$ represents the mixing probability. The dataset $\mathcal{D}_{0.0}$ corresponds to data generated from the same distribution as the training dataset (but with new samples), allowing for an evaluation of the models on unseen, unshifted data. Conversely, $\mathcal{D}_{1.0}$ corresponds to the fully shifted dataset. Intermediate values of $p$ represent datasets with increasing proportions of shifted data, enabling a systematic investigation of model robustness under varying degrees of covariate distribution shift.
