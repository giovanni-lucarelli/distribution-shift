\chapter{Performance Enhancement}
\chapterauthor{Giacomo Amerio - Andrea Spinelli}

One of the most used approach to mitigate covariate shift consequences is \textit{Reweighting}, which consists in quantify the degree of distribution shift and then apply a correction to the model \cite{zhang}. Another approach is \textit{Data Augmentation}, which consists in generating new data points from the original ones, in order to make the model more robust to the distribution shift \cite{zhao}. In this chapter, we introduce an innovative approach we term \textbf{\textit{Random Augmentation Walk}}. In particular, we will show how applying this pre-processing method to the training step of a Gradient Boosting model, leads to an improvement in performances for both Classification and Regression tasks.

% \section{Overfitting}


% The simplest approach to improving model performance is to overfit the model on the training data. The idea is that by overfitting to the training data, the model may better capture the underlying data distribution, potentially leading to improved performance on the shifted test data—particularly for data points that lie within the intersection of the training and test distributions.

% In order to overfit the models, we created a custom script to perform grid search withot cross-validation. We favored the hyperparameters that achieved the highest score on the training set, while ignoring the necessary precaution to avoid overfitting, i.e. we overshot the maximum depth reachable by the trees.

% %Since the logistic GAM is a method which automatically selects the functions that best fit the data, we did not perform any overfitting on it. 
% Since the logistic GAM uses different methods to perform fine tuning from the other ones, we are still trying to figure out how to implement a proper overfit. The results for the other models are shown in \cref{fig:overfitted-models-perf}.

% \begin{figure}[H]
%     \centering
%     \includegraphics[width=0.8\textwidth]{assets/overfit.png} 
%     \caption{\textbf{Performance comparison of overfitted models under covariate shift}}
%     \label{fig:overfitted-models-perf}
% \end{figure}


% % Although the GAM does not appear to be suffering from overfitting,
% As depicted in \cref{fig:overfitted-models-perf}, the performance of the models declines as the percentage of mixed data points increases, falling short of the levels achieved by the fine-tuned models. The plot also highlights that overfitting resulted in greater variability in the models' AUC scores. Among the models, the decision tree was the most adversely affected by the overfitting process, whereas the logistic regression model demonstrated relatively better performance compared to the extreme gradient boosting model.

\section{Random Augmentation Walk}

This method is based on the idea of \textbf{Data Augmentation}. Instead of using training data as it is, we generate new data applying the following transformation to the original dataset:

% \begin{algorithm}[H]
%     \caption{Random Augmentation Walk}
%     \begin{algorithmic}[1]
%         \Statex \textbf{Input:} $Data_\text{train}$,$Size$, $N$, $\varepsilon$
%         \Statex \textbf{Output:} $Data_\text{final}$
%         \Statex
%         \State $Size$ \leftarrow $len(Data)$ 
%         \State $Data_\text{new}$ \leftarrow $Data$
%         \State $Data_\text{tr}$ \leftarrow random subset of $N\%$ of $Data$
%         \For{$x_i$ in $Data_\text{tr}$}
%             \State $x_i' \leftarrow 
%             \begin{cases}
%                 X_i + \varepsilon & \text{with probability } 0.5 \\
%                 X_i - \varepsilon & \text{with probability } 0.5
%             \end{cases}$
%             \State $y_i' \leftarrow y_i$
%         \EndFor
%         \State $Data_\text{aug} \leftarrow Data_\text{new} \cup Data_\text{tr}$
%         \State $Data_\text{aug} \leftarrow Downsample(Data_\text{aug}, Size)$
%         \State\Return $Data_\text{aug}$
%     \end{algorithmic}
% \end{algorithm}


\begin{algorithm}[H]
    
        \begin{algorithmic}[1]
            \STATE \textbf{Input:} $Data_{\text{train}}$, $Size$, $N$, $\varepsilon$.\\[0.5em]
         
            \State $Data_{\%}$ \leftarrow $random subset of N\% of Data_{train}$\\[0.5em]
                \textbf{For} $x_i$ in Data_{\%}\\[0.5em]
                    \State \phantom{mm} $x_i' \leftarrow 
                    \begin{cases}
                        X_i + \varepsilon & \text{with probability } 0.5 \\
                        X_i - \varepsilon & \text{with probability } 0.5
                    \end{cases}$\\[0.5em]
                    \State \phantom{mm} $y_i' \leftarrow y_i$\\[0.5em]
                \textbf{End For}\\
                \State $Data_\text{aug}$ \leftarrow $Data_{train} \cup Data_{\%}$\\
                \State $Data_\text{final}$ \leftarrow $Downsample(Data_\text{aug}, Size)$\\
            \STATE \textbf{Return} $Data_{\text{final}}$
        \end{algorithmic}
        \caption{Let $Data_{\text{train}}$ represent the training dataset, $\text{Size}$ denote the size of $Data_{\text{train}}$ , $N$ specify the percentage of data to be augmented, and $\varepsilon$ define the magnitude of the applied shift. The parameter $\varepsilon$ is a constant determined a posteriori through a grid search over a predefined range of possible values. The direction of the shift is randomly selected.}
\end{algorithm}



Interestingly, this method does not require any knolewdge of the shifted test distributions, it just performs a noising step on a variable percentage of the training data. Then it downsamples the augmented data to the original size.
It is important to note that despite the variation in the $x_i'$ values, the $y_i$ values remain the same.

\subsection{Experiments}
\subsubsection{\textbf{Classification Task}}

We evaluate the impact of the R.A.W. method on the same binary classification task illustrated in chapter 3. The experiment is conducted as follows:
\subsubsection{Training Set}
The training set consists of 10,000 data points with 3 features and 1 binary target variable. The data points are generated using a multivariate normal distribution with the following parameters:

\begin{itemize}
    \item $ \boldsymbol{X}_{\text{train}} = (X_{\text{train}\,1}, X_{\text{train}\,2}, X_{\text{train}\,3}) \sim \mathcal{N}(\boldsymbol{\mu}_{\text{train}}, \boldsymbol{\Sigma}_{\text{train}}) $
    \item $ \mu_{\text{train}\,i} \sim \mathcal{U}_{[0,1]} $ for $ i = 1, 2, 3 $
    \item $ [\boldsymbol{\Sigma}_{\text{train}}]_{i,j} \sim \mathcal{U}_{[-1,1]} $ for $ i, j = 1, 2, 3 $
\end{itemize}

Meanwhile, the target variable is generated using the following procedure:

\begin{enumerate}
    \item $$ 
    z = \beta_0 + \sum_{i=1}^3 \beta_i x_i + \sum_{i=1}^3 \beta_{ii} x_i^2 + \sum_{i=1}^{2} \sum_{j=i+1}^3 \beta_{ij} x_i x_j\,,   \quad \beta_{\cdot} \sim \mathcal{U}_{[-1,1]}
    $$
    \item $$ p = \frac{1}{1 + e^{-z}}$$
    \item $$ Y \sim \text{Be}(p)$$
\end{enumerate}

\subsubsection{Test Sets}
The test sets are generated in the same way as the training set, but with the following modifications:

\begin{itemize}
    \item $ \boldsymbol{X}_{\text{shift}} = (X_{\text{shift}\,1}, X_{\text{shift}\,2}, X_{\text{shift}\,3}) \sim \mathcal{N}(\boldsymbol{\mu}_{\text{shift}}, \boldsymbol{\Sigma}_{\text{shift}}) $
    \item $ \boldsymbol{\mu}_{\text{shift}} = \mathcal{Q}_{0.05}(\boldsymbol{X}_{\text{train}})$
    \item $ [\boldsymbol{\Sigma}_{\text{shift}}]_{i,j} \sim \mathcal{U}_{[-0.5,0.5]} $ for $ i, j = 1, 2, 3 $
\end{itemize}
Each model is evaluated on 11 distinct test sets, where each test set is generated with a varying percentage of shifted data points, representing different statistical mixtures.

The models' performance is simulated and assessed across 50 instances of 11 distinct statistical mixtures, ensuring a sufficient number of trials to rigorously validate significance using two samples Student's t-Test.

\subsubsection{Models}









\subsubsection{\textbf{Regression Task}}

A 1-dimensional regression problem is considered to evaluate the performance of the Random Augmentation Walk method. The experiment is conducted as follows:

\subsection{Training Set}
The training set consists of 10,000 data points generated with $x$ values linearly spaced between -3 and 3, and $y$ values generated with the following formula:
\begin{equation}
    y = \sin(x)\exp(-x^2) + \varsigma
\end{equation}

Where $\varsigma$ is a random noise term drawn from a normal distribution with mean 0 and standard deviation 0.1.

\begin{figure}[H]
    \centering
    \includegraphics[width=0.8\textwidth]{assets/fit_on_train.png} 
    \caption{\textbf{Baseline and augmented models fit on the training data} To note that the augmented model has a \textbf{looser fit} on the training data.}
    \label{fig:fit-train}
\end{figure}

\subsection{Test Sets}
Thirty test sets are created by shifting the $x$ values by factors ranging from 1.5 to 5.5. Each shifted test set is generated independently using the same underlying function and noise process as the training data.
\begin{figure}[H]
    \centering
    \includegraphics[width=0.8\textwidth]{assets/reg_shift_plot.png} 
    \caption{\textbf{A handful of test sets depicted together with the training set}}
    \label{fig:reg-shift-plot}
\end{figure}

\subsection{Models}
Two instances of the same model are trained with and without the Random Augmentation Walk method. The models are configured as follows:
\begin{itemize}
    \item \textbf{Baseline Model}: A Gradient Boosting Regressor (GBR) is employed as a baseline model. The GBR is configured with the following hyperparameters: \plaintt{n\_estimators=100, max\_depth=5, learning\_rate=0.05}.
    \item \textbf{R.A.W. Model}: This model has the same features as the baseline GBR but leverages the key parameters of the custom data augmentation method. The percentage of training data (\plaintt{fraction\_to\_shift}) to be augmented is set to 40\%. Meanwhile the \plaintt{base\_shift\_factor} controls the magnitude of shifts considered in training (set to 1 in this experiment).
\end{itemize}




\subsection{\textbf{Results}}

\subsubsection{\textbf{Classification Task}}









\subsubsection{\textbf{Regression Task}}

\subsubsection{Evaluation Metric}
The model performance is evaluated using the Mean Squared Error (MSE). For each shifted test set, the MSE is computed for both the baseline and augmented models.
The improvement is defined as the relative reduction in MSE computed as :
\begin{equation}
    \text{Improvement} = \left(\frac{\text{MSE}_{\text{baseline}} - \text{MSE}_{\text{aug}}}{\text{MSE}_{\text{baseline}}}\right) \times 100\%
\end{equation}

The metric is computed for each shift, and the results are shown in the figure below.
\begin{figure}[H]
    \centering
    \includegraphics[width=0.8\textwidth]{assets/reg_exp_improvement.png} 
    \caption{\textbf{Model Improvement over Shifted Test Sets.} The red dotted line is the mean improvement across all shifts.}
    \label{fig:improv-plot}
\end{figure}

As we can see from the plot, the augmented model has worse performance then the baseline model for relatively small shifts but as the shift in data points becomes more significant, the augmented model outperforms the baseline model. The mean improvement across all shifts is still positive.
We believe the promising results of this methods are still to be analyzed in depth, but the preliminary results are encouraging.
